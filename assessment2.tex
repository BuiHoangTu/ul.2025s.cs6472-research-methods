\documentclass[a4paper,12pt]{article}
\usepackage{graphicx}
\usepackage{hyperref}
\usepackage{setspace}
\usepackage{enumitem}
\usepackage{titlesec}
\usepackage{times}
\usepackage{longtable}
\usepackage{apacite}


\begin{document}

\begin{center}
    \textbf{\Large Assessment 2} \vspace{0.5cm}
\end{center}

\textbf{Name: Hoang Tu Bui} \\
\textbf{Student ID: 24005665} 

\section{ Research Question (10 marks)}
In large-scale wind farms, how does a CNN autoencoder-based Normal Behaviour Monitoring Model compare to conventional threshold-based monitoring techniques and other normal behavior models (such as long short-term memory) in terms of accuracy and false positive rate for anomaly detection?

\section{Search Query (5 marks)}
Here are queries that I used to search for articles related to the research question:
\begin{itemize}
    \item wind turbine AND (normal behavior monitoring OR anomaly detection)
    \item "(CNN OR LSTM) AND autoencoder"
\end{itemize}

\section{Mapping of Articles (15 marks)}
% Select four research articles relevant to your research question, ensuring they belong to at least two different categories. Then, create a structured mapping of these articles using a format of your choice. You may present this as a table, a list, or a mind map, whichever best suits you.

The mapping of the articles is presented in Table \ref{tab:article_mapping}.

\begin{longtable}{
    |p{0.2\linewidth}
    |p{0.13\linewidth}
    |p{0.1\linewidth}
    |p{0.2\linewidth}
    |p{0.2\linewidth}
    |p{0.13\linewidth}|
}
    \caption{Mapping of Articles} 
    \label{tab:article_mapping} \\
    
    \hline
    \textbf{Article} & 
    \textbf{Categories} &
    \textbf{Dataset} &
    \textbf{Methodology} &
    \textbf{Findings} &
    \textbf{Limitations} \\
    \hline
    \endfirsthead
    
    \hline
    \textbf{Article} & 
    \textbf{Categories} &
    \textbf{Dataset} &
    \textbf{Methodology} &
    \textbf{Findings} &
    \textbf{Limitations} \\
    \hline
    \endhead
    
    \hline
    \endfoot
    
    \hline
    \endlastfoot
    
    Analytical investigation of autoencoder-based methods for unsupervised anomaly detection in building energy data \cite{fan_analytical_2018} &
    Framework &
    Private dataset &
    Compare simple CNN and RNN performance with different augmentation techniques &
    Proposed adding noise to the train data, methods to evaluate the AE performance &
    Tested models are small, dataset is closely related to wind turbine \\
    \hline
    
    Anomaly detection and fault analysis of wind turbine components based on deep learning network \cite{zhao_anomaly_2018} & 
    Multi-phrase training & 
    SCADA & 
    Pre-train the RBM model unsupervised then add supervised layers for fine tuning using labels & 
    Adaptive thresholds derived from extreme value theory helps reduce false positive. The model is 20\% faster in term of inference time and 60\% faster in term of early predicting (10h) &
    The compared NN model is simple FCNN \\
    \hline
    
    Autoencoder-based Condition Monitoring and Anomaly Detection Method for Rotating Machines \cite{ahmad_autoencoder-based_2020}&
    LSTM, Framework & 
    NASA bearing dataset & 
    Use LSTM AE to extract features and then use Gaussian model and Mahalanobis distance to determine if abnormal &
    Use Isolation Forest model to evaluate high level features extracted (vs base data). High F1 score of 0.996 & 
    Does not show/describe the model structure. The dataset is not closely related \\
    \hline
    
    Transfer learning applications for autoencoder-based anomaly detection in wind turbines \cite{roelofs_transfer_2024} & 
    Multi-phrase training & 
    SCADA & 
    Train model on 1-year of common turbines data then tune the model by 1, 2 or 3-month data of the target data &
    For transfer learning, tuning the model threshold is sufficient. Tuning the whole model or just the decoder only brings considerable improvement & 
    The models tested are simple FCNN model \\
    \hline

    Wind Turbine Anomaly Detection Using Mahalanobis Distance and SCADA Alarm Data \cite{liu_wind_2022} &
    Machine learning &
    SCADA &
    First, use FCNN as autoencoder to extract features, then use Mahalanobis distance to determine if abnormal &
    The model can explain the anomaly in each component &
    The model require professional knowledge to determine the threshold \\
    \hline

\end{longtable}

\section{Literature Survey (60 marks)}
% Write a literature survey based on the five selected articles for your research question. The literature review should be \textbf{1000-1500 words} and should analyze and synthesize the findings from the articles, highlighting key themes, methodologies, and gaps in the existing research.

The development of reliable anomaly detection systems for large-scale wind farms has become increasingly critical as wind energy plays a greater role in global power generation. Traditional statistic-based monitoring techniques, while conceptually simple, suffer from high false positive rates and require substantial domain expertise for threshold calibration \cite{liu_wind_2022}. These limitations have driven research into autoencoder-based normal behavior models that can learn complex patterns from operational data without requiring labeled fault examples. Current approaches predominantly explore two architectural paradigms: convolutional neural network (CNN) autoencoders that excel at capturing spatial relationships in data, and recurrent architectures like long short-term memory (LSTM) networks that model temporal dependencies. The comparative effectiveness of these approaches remains an open research question, particularly regarding their accuracy and false positive rates in operational wind farm environments.

Foundational work by \citeauthor{fan_analytical_2018} established important methodological frameworks for evaluating autoencoder performance, though their focus on building energy data requires adaptation for wind turbine applications. Their comparative study of CNN and RNN autoencoders demonstrated that adding specific noise types (additive isotropic Gaussian, salt-and-pepper, or masking noise) during training could significantly improve model robustness. Perhaps more importantly, they developed innovative evaluation metrics that could inform wind turbine research, including analyzing the normality of reconstruction residual distributions and assessing extracted features by predicting auxiliary variables like time-of-day indicators. While their models were relatively simple, their pipeline for training and evaluating autoencoders provides valuable methodological guidance, particularly their finding that convolutional architectures showed particular promise for certain data relationships.

For direct wind turbine applications, \citeauthor{zhao_anomaly_2018} presented a sophisticated deep autoencoder approach using stacked Restricted Boltzmann Machines (RBMs) that addressed several key challenges. Their two-phase training process first employed unsupervised pre-training with RBMs followed by supervised fine-tuning with labeled data, combining the benefits of both learning paradigms. Most notably, they introduced an adaptive thresholding method based on extreme value theory that reduced false positives by 20\% compared to fixed thresholds. Their results demonstrated substantial practical advantages, including detecting anomalies 10 hours earlier than conventional neural network approaches while requiring only 20\% of the prediction time. However, their use of RBM-based architectures rather than CNNs or LSTMs leaves open questions about whether these alternative architectures might better capture the spatiotemporal dependencies in SCADA data. Their success with adaptive thresholding nevertheless suggests an important supplementary technique that could benefit other autoencoder variants.

The temporal nature of wind turbine operational data has motivated several LSTM-based approaches. \citeauthor{ahmad_autoencoder-based_2020} achieved exceptional performance (F1-score 99.6\%) on rotating machinery data using an LSTM autoencoder combined with Gaussian models and Mahalanobis distance for anomaly scoring. Their innovative evaluation method using Isolation Forest to assess the quality of extracted high-level features provides a valuable alternative to reconstruction error analysis. However, the limited model architecture details in their publication and their use of NASA bearing data rather than widely used public SCADA data make direct comparisons difficult. Their results do suggest that LSTM architectures merit serious consideration for wind turbine applications, particularly when combined with statistical post-processing techniques like Mahalanobis distance. The high computational cost of LSTMs versus simpler architectures remains an important practical consideration for large-scale deployment.

Practical training challenges in wind farms have encouraged researchers into transfer learning solutions. \citeauthor{roelofs_transfer_2024} conducted comprehensive experiments comparing different transfer learning strategies for autoencoder-based anomaly detection across multiple turbines. Their finding that models trained on multiple source turbines showed no improvement over single-source models due to potential overfitting has important implications for model development. Surprisingly, their results indicated that simply adjusting the detection threshold during transfer could achieve comparable performance to more extensive fine-tuning, though minor improvements were possible by fine-tuning the decoder or entire autoencoder. These findings suggest that relatively simple transfer learning approaches may be sufficient for many practical scenarios, though their study was limited to fully connected architectures. The potential for applying similar transfer learning strategies to CNN or LSTM autoencoders remains an important area for future research.

\citeauthor{liu_wind_2022} presented a contrasting approach using a simple fully connected autoencoder with Mahalanobis distance for anomaly detection. Their method's strong explainability and low computational requirements make it particularly appealing for operational deployment, though it still requires expert knowledge for threshold determination. The success of their Mahalanobis distance approach echoes similar findings by \citeauthor{ahmad_autoencoder-based_2020}, suggesting this statistical technique may provide a robust complement to various autoencoder architectures. However, the relatively simple network architecture in Liu's study leaves unanswered whether more sophisticated models could achieve better accuracy while maintaining explainability.

Several critical research gaps emerge from this synthesis. First, direct comparisons between CNN and LSTM autoencoders using identical wind turbine SCADA data and evaluation metrics are conspicuously absent from the literature. Most existing studies compare against overly simplistic baselines rather than contemporary alternatives. Second, while multiple studies have demonstrated the effectiveness of supplementary techniques like adaptive thresholding \cite{zhao_anomaly_2018} and Mahalanobis distance \cite{ahmad_autoencoder-based_2020}, their relative merits and potential synergies remain unexplored. Third, the transfer learning strategies shown effective by \citeauthor{roelofs_transfer_2024} for simple architectures need validation with more complex CNN and LSTM models. Finally, practical implementation factors like computational efficiency, model maintenance, and explainability require greater attention, particularly for large-scale wind farm deployment.

The reviewed literature collectively suggests that hybrid approaches combining deep autoencoders with statistical post-processing may offer the most promising direction. CNN architectures appear particularly suited to capturing spatial relationships in SCADA parameters, while LSTMs may better model temporal operational patterns. Adaptive thresholding techniques like those developed by \citeauthor{zhao_anomaly_2018} could help maintain low false positive rates, while transfer learning strategies from \citeauthor{roelofs_transfer_2024} may enable practical deployment across diverse turbines. Future research should prioritize comprehensive benchmarking studies that control for data quality and evaluation metrics while systematically comparing architectural choices and supplementary techniques. The development of standardized evaluation protocols and publicly available benchmark datasets would significantly advance the field, enabling more rigorous comparison of approaches and accelerating progress toward reliable, large-scale wind turbine monitoring systems.


\section{References (10 marks)}
% Ensure proper in-text citations and include a bibliography. You are free to use any recognized citation format (e.g., APA, MLA, Harvard, IEEE), but it must be consistent throughout your article.

\bibliographystyle{apacite}
\bibliography{assessment2.bib}

\end{document}