\documentclass[a4paper,12pt]{article}
\usepackage{graphicx}
\usepackage{hyperref}
\usepackage{setspace}
\usepackage{enumitem}
\usepackage{titlesec}
\usepackage{times}
\usepackage{longtable}
\usepackage{apacite}


\begin{document}

\begin{center}
    \textbf{\Large Assessment 2} \vspace{0.5cm}
\end{center}

\textbf{Name: Hoang Tu Bui} \\
\textbf{Student ID: 24005665} 

\section{ Research Question (10 marks)}
In large-scale wind farms, how does a CNN autoencoder-based Normal Behaviour Monitoring Model compare to conventional threshold-based monitoring techniques and other normal behavior models (such as long short-term memory) in terms of accuracy and false positive rate for anomaly detection?

\section{Search Query (5 marks)}
Here are queries that I used to search for articles related to the research question:
\begin{itemize}
    \item wind turbine AND (normal behavior monitoring OR anomaly detection)
    \item "(CNN OR LSTM) AND autoencoder"
\end{itemize}

\section{Mapping of Articles (15 marks)}
% Select four research articles relevant to your research question, ensuring they belong to at least two different categories. Then, create a structured mapping of these articles using a format of your choice. You may present this as a table, a list, or a mind map, whichever best suits you.

The mapping of the articles is presented in Table \ref{tab:article_mapping}.

\begin{longtable}{
    |p{0.2\linewidth}
    |p{0.13\linewidth}
    |p{0.1\linewidth}
    |p{0.2\linewidth}
    |p{0.2\linewidth}
    |p{0.13\linewidth}|
}
    \caption{Mapping of Articles} 
    \label{tab:article_mapping} \\
    
    \hline
    \textbf{Article} & 
    \textbf{Categories} &
    \textbf{Dataset} &
    \textbf{Methodology} &
    \textbf{Findings} &
    \textbf{Limitations} \\
    \hline
    \endfirsthead
    
    \hline
    \textbf{Article} & 
    \textbf{Categories} &
    \textbf{Dataset} &
    \textbf{Methodology} &
    \textbf{Findings} &
    \textbf{Limitations} \\
    \hline
    \endhead
    
    \hline
    \endfoot
    
    \hline
    \endlastfoot
    
    Analytical investigation of autoencoder-based methods for unsupervised anomaly detection in building energy data \cite{fan_analytical_2018} &
    Framework &
    Private dataset &
    Compare simple CNN and RNN performance with different augmentation techniques &
    Proposed adding noise to the train data, methods to evaluate the AE performance &
    Tested models are small, dataset is closely related to wind turbine \\
    \hline
    
    Anomaly detection and fault analysis of wind turbine components based on deep learning network \cite{zhao_anomaly_2018} & 
    Multi-phrase training & 
    SCADA & 
    Pre-train the RBM model unsupervised then add supervised layers for fine tuning using labels & 
    Adaptive thresholds derived from extreme value theory helps reduce false positive. The model is 20\% faster in term of inference time and 60\% faster in term of early predicting (10h) &
    The compared NN model is simple FCNN \\
    \hline
    
    Autoencoder-based Condition Monitoring and Anomaly Detection Method for Rotating Machines \cite{ahmad_autoencoder-based_2020}&
    LSTM, Framework & 
    NASA bearing dataset & 
    Use LSTM AE to extract features and then use Gaussian model and Mahalanobis distance to determine if abnormal &
    Use Isolation Forest model to evaluate high level features extracted (vs base data). High F1 score of 0.996 & 
    Does not show/describe the model structure. The dataset is not closely related \\
    \hline
    
    Transfer learning applications for autoencoder-based anomaly detection in wind turbines \cite{roelofs_transfer_2024} & 
    Multi-phrase training & 
    SCADA & 
    Train model on 1-year of common turbines data then tune the model by 1, 2 or 3-month data of the target data &
    For transfer learning, tuning the model threshold is sufficient. Tuning the whole model or just the decoder only brings considerable improvement & 
    The models tested are simple FCNN model \\
    \hline

    Wind Turbine Anomaly Detection Using Mahalanobis Distance and SCADA Alarm Data \cite{liu_wind_2022} &
    Machine learning &
    SCADA &
    First, use FCNN as autoencoder to extract features, then use Mahalanobis distance to determine if abnormal &
    The model can explain the anomaly in each component &
    The model require professional knowledge to determine the threshold \\
    \hline

\end{longtable}

\section{Literature Survey (60 marks)}
% Write a literature survey based on the five selected articles for your research question. The literature review should be \textbf{1000-1500 words} and should analyze and synthesize the findings from the articles, highlighting key themes, methodologies, and gaps in the existing research.

The increasing global adoption of wind energy has created a pressing need for more reliable condition monitoring systems, particularly for large-scale wind farms where manual inspection is impractical. Traditional approaches relying on threshold-based monitoring techniques have shown significant limitations, often producing high false positive rates and requiring substantial expert knowledge for threshold determination \cite{liu_wind_2022}. These conventional methods struggle particularly with the complex, non-linear relationships in SCADA data and the extreme value distributions common in wind turbine operations \cite{zhao_anomaly_2018}. Recent advances in deep learning have introduced more sophisticated approaches, particularly autoencoder-based normal behavior monitoring models that learn healthy operational patterns and detect deviations without requiring labeled fault data. Among these, convolutional neural network (CNN) autoencoders and long short-term memory (LSTM) networks have emerged as promising solutions, though their comparative effectiveness remains under-explored in the context of wind turbine anomaly detection.

Current research demonstrates both the potential and limitations of various autoencoder architectures. \citeauthor{fan_analytical_2018} provided foundational insights into autoencoder performance through their comparative study of different architectures, though their focus on building energy data rather than wind turbines limits direct applicability. Their finding that adding noise to training data improves model robustness may nevertheless inform wind turbine applications. More specifically, \citeauthor{zhao_anomaly_2018} developed a deep autoencoder using restricted Boltzmann machines (RBMs) for wind turbine component monitoring, reporting a 60\% improvement in early fault prediction (10 hours ahead) compared to simple fully connected neural networks, along with a 20\% reduction in false positives through their adaptive threshold approach based on extreme value theory. However, their study's relatively simple baseline comparison leaves open questions about how more sophisticated architectures might perform.

The choice between CNN and LSTM architectures presents an important research question, as each has distinct advantages for processing different types of operational data. \citeauthor{ahmad_autoencoder-based_2020} demonstrated the effectiveness of LSTM autoencoders for rotating machinery, achieving an exceptional F1-score of 0.996 when combined with Gaussian models and Mahalanobis distance, though their use of NASA bearing data rather than wind turbine SCADA data limits direct comparability. Meanwhile, \citeauthor{liu_wind_2022} employed a simpler fully connected autoencoder with Mahalanobis distance for wind turbine anomaly detection, showing promising results but still requiring expert knowledge for threshold interpretation. This suggests that while various autoencoder approaches show improvement over conventional methods, the optimal architecture and implementation strategy for wind turbine monitoring remains unclear.

A critical challenge across autoencoder implementations is determining appropriate thresholds for anomaly detection. Both \citeauthor{liu_wind_2022} and \citeauthor{ahmad_autoencoder-based_2020} employed Mahalanobis distance on extracted features with success, while \citeauthor{zhao_anomaly_2018} demonstrated the effectiveness of extreme value theory for adaptive thresholding. These approaches help address the high false positive rates that plague simpler methods, but their comparative effectiveness warrants further investigation. Recent work by \citeauthor{roelofs_transfer_2024} on transfer learning for autoencoder-based anomaly detection provides valuable insights, though their focus on simple fully connected networks rather than CNN or LSTM architectures leaves open questions about how these more complex models might perform in transfer learning scenarios. Their finding that threshold adjustment alone can provide substantial benefits during model transfer has important implications for practical implementation across diverse wind farms.

Several key gaps emerge from the current literature. First, there is limited direct comparison between CNN autoencoders and LSTM-based approaches specifically for wind turbine anomaly detection under identical conditions. Most existing studies compare against relatively simple baseline models rather than conducting comprehensive benchmarking against state-of-the-art methods. Additionally, few papers address the computational efficiency and real-time applicability of these models in large-scale wind farm environments, despite the practical importance of these factors for operational deployment. The transfer learning approaches explored by \citeauthor{roelofs_transfer_2024} for simple networks remain underexplored for more complex architectures like CNN and LSTM autoencoders. \citeauthor{chesterman_overview_2023} provide a broader overview of normal behavior modeling approaches that highlights these limitations, particularly the challenges of limited training data and model maintenance in practical settings.

In conclusion, while autoencoder-based approaches clearly outperform conventional threshold-based methods in terms of accuracy and false positive rates, the current literature lacks definitive evidence about the relative merits of CNN versus LSTM architectures for wind turbine monitoring. The existing research demonstrates that both can be effective, but direct comparisons using identical datasets and evaluation metrics are needed. Future work should also address the practical implementation challenges in operational wind farms, including computational requirements, model maintenance, and the development of more automated threshold determination methods. The promising results from studies like \citeauthor{zhao_anomaly_2018} and \citeauthor{ahmad_autoencoder-based_2020} suggest that hybrid approaches combining deep learning with statistical methods like extreme value theory or Mahalanobis distance may offer the most robust solutions, but this hypothesis requires systematic validation through carefully designed comparative studies.


\section{References (10 marks)}
% Ensure proper in-text citations and include a bibliography. You are free to use any recognized citation format (e.g., APA, MLA, Harvard, IEEE), but it must be consistent throughout your article.

\bibliographystyle{apacite}
\bibliography{assessment2.bib}

\end{document}