\documentclass[a4paper,12pt]{article}
\usepackage{graphicx}
\usepackage{hyperref}
\usepackage{setspace}
\usepackage{enumitem}
\usepackage{titlesec}
\usepackage{times}
\usepackage{longtable}
\usepackage{apacite}


\begin{document}

\begin{center}
    \textbf{\Large Assessment 2} \vspace{0.5cm}
\end{center}

\textbf{Name: Hoang Tu Bui} \\
\textbf{Student ID: 24005665} 

\section{ Research Question (10 marks)}
In large-scale wind farms, how does a CNN autoencoder-based Normal Behaviour Monitoring Model compare to conventional threshold-based monitoring techniques and other normal behavior models (such as long short-term memory) in terms of accuracy and false positive rate for anomaly detection?

\section{Search Query (5 marks)}
Here are queries that I used to search for articles related to the research question:
\begin{itemize}
    \item wind turbine AND (normal behavior monitoring OR anomaly detection)
    \item "(CNN OR LSTM) AND autoencoder"
\end{itemize}

\section{Mapping of Articles (15 marks)}
% Select four research articles relevant to your research question, ensuring they belong to at least two different categories. Then, create a structured mapping of these articles using a format of your choice. You may present this as a table, a list, or a mind map, whichever best suits you.

The mapping of the articles is presented in Table \ref{tab:article_mapping}.

\begin{longtable}{
    |p{0.2\linewidth}
    |p{0.13\linewidth}
    |p{0.1\linewidth}
    |p{0.2\linewidth}
    |p{0.2\linewidth}
    |p{0.13\linewidth}|
}
    \caption{Mapping of Articles} 
    \label{tab:article_mapping} \\
    
    \hline
    \textbf{Article} & 
    \textbf{Categories} &
    \textbf{Dataset} &
    \textbf{Methodology} &
    \textbf{Findings} &
    \textbf{Limitations} \\
    \hline
    \endfirsthead
    
    \hline
    \textbf{Article} & 
    \textbf{Categories} &
    \textbf{Dataset} &
    \textbf{Methodology} &
    \textbf{Findings} &
    \textbf{Limitations} \\
    \hline
    \endhead
    
    \hline
    \endfoot
    
    \hline
    \endlastfoot
    
    Analytical investigation of autoencoder-based methods for unsupervised anomaly detection in building energy data \cite{fan_analytical_2018} &
    Framework &
    Private dataset &
    Compare simple CNN and RNN performance with different augmentation techniques &
    Proposed adding noise to the train data, methods to evaluate the AE performance &
    Tested models are small, dataset is closely related to wind turbine \\
    \hline
    
    Anomaly detection and fault analysis of wind turbine components based on deep learning network \cite{zhao_anomaly_2018} & 
    Multi-phrase training & 
    SCADA & 
    Pre-train the RBM model unsupervised then add supervised layers for fine tuning using labels & 
    Adaptive thresholds derived from extreme value theory helps reduce false positive. The model is 20\% faster in term of inference time and 60\% faster in term of early predicting (10h) &
    The compared NN model is simple FCNN \\
    \hline
    
    Autoencoder-based Condition Monitoring and Anomaly Detection Method for Rotating Machines \cite{ahmad_autoencoder-based_2020}&
    LSTM, Framework & 
    NASA bearing dataset & 
    Use LSTM AE to extract features and then use Gaussian model and Mahalanobis distance to determine if abnormal &
    Use Isolation Forest model to evaluate high level features extracted (vs base data). High F1 score of 0.996 & 
    Does not show/describe the model structure. The dataset is not closely related \\
    \hline
    
    Transfer learning applications for autoencoder-based anomaly detection in wind turbines \cite{roelofs_transfer_2024} & 
    Multi-phrase training & 
    SCADA & 
    Train model on 1-year of common turbines data then tune the model by 1, 2 or 3-month data of the target data &
    For transfer learning, tuning the model threshold is sufficient. Tuning the whole model or just the decoder only brings considerable improvement & 
    The models tested are simple FCNN model \\
    \hline

    Wind Turbine Anomaly Detection Using Mahalanobis Distance and SCADA Alarm Data \cite{liu_wind_2022} &
    Machine learning &
    SCADA &
    First, use FCNN as autoencoder to extract features, then use Mahalanobis distance to determine if abnormal &
    The model can explain the anomaly in each component &
    The model require professional knowledge to determine the threshold \\
    \hline

\end{longtable}

\section{Literature Survey (60 marks)}
% Write a literature survey based on the five selected articles for your research question. The literature review should be \textbf{1000-1500 words} and should analyze and synthesize the findings from the articles, highlighting key themes, methodologies, and gaps in the existing research.

\section{References (10 marks)}
% Ensure proper in-text citations and include a bibliography. You are free to use any recognized citation format (e.g., APA, MLA, Harvard, IEEE), but it must be consistent throughout your article.

\bibliographystyle{apacite}
\bibliography{assessment2.bib}

\end{document}